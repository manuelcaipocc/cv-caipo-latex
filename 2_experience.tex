\begin{tabular}{@{}>{\raggedleft\arraybackslash}m{\datecol} @{\hspace{\colgap}} p{\dimexpr\textwidth-\datecol-\colgap\relax}@{}}


% ===== Bosch =====
\datecell[-1.6ex]{\small 11.2024 – Heute} &
\begin{minipage}[t]{\linewidth}\vspace{0pt}
\begin{expblock}
  \footnotesize
  \setlength{\parindent}{0pt}\setlength{\parskip}{0pt} % texto más pegado
  \textbf{\href{https://www.boschrexroth.com/en/dc/}{Bosch Rexroth, Ulm, Deutschland}}\\[-0.1em]
  % \textbf{Bosch Rexroth, Ulm, Deutschland}\\[-0.1em]
  \textbf{SSD3 – Data Scientist Werkstudent}\\[-1em]
  \begin{itemize}[nosep,leftmargin=1em] % listas más compactas
    \item Aufbau und Orchestrierung von ETL- und Datenpipelines zur Verarbeitung hochfrequenter Zeitreihendaten (Relationalen \& NoSQL-Datenbanken).
    \item Integration von Message-Broker-Systemen (Solace, MQTT) und Optimierung des industriellen Datenflusses für skalierbare Anwendungen.
    \item Entwicklung eines Überwachungssystems für Hydraulikanlagen mit Einsatz von ML- \& DL-Methoden zur Erkennung anomaler Betriebszustände und zur Vorhersage der Restlebensdauer (RUL).
    \item Sicherstellung von Transparenz \& Akzeptanz durch den Einsatz erklärbarer KI-Ansätze in industriellen Prognosemodellen.
  \end{itemize}
\end{expblock}
\end{minipage}
\\[-0.2em] % menos espacio entre filas

% ===== Freeport =====
\datecell[-1.6ex]{\small 04.2021 – 07.2023} &
\begin{minipage}[t]{\linewidth}\vspace{0pt}
\begin{expblock}
  \footnotesize
  \setlength{\parindent}{0pt}\setlength{\parskip}{0pt}
  \textbf{\href{https://www.fcx.com}{Freeport-McMoRan Cerro Verde, Phoenix, U.S.A.}}\\[-0.1em]
  \textbf{Data Analytics – Junior Data Scientist 2}\\[-1em]
\begin{itemize}[nosep,leftmargin=1em]
  \item Entwicklung und Skalierung von ML/DL-Modellen (LSTM, CNN, XGBoost) in Azure ML für Anomalieerkennung und Verschleißprognosen; weltweiter Einsatz in mehreren Minen.
  \item Implementierung von Explainability (SHAP, PDP) und kontinuierlichen Retraining-Prozessen zur transparenten, adaptiven Modellnutzung.
  \item Einführung von RUL-Modellen zur proaktiven Wartungsplanung; +1,5 \% Anlagendisponibilität (≈10 Mio. USD/Jahr Einsparung). Ausgezeichnet mit Innova 2022, Showroom Award und President’s Award.
\end{itemize}


  \textbf{Data Analytics – Junior Data Scientist 1}\\[-1em]
  \begin{itemize}[nosep,leftmargin=1em]
    \item Implementierung von ML-Modellen in Azure ML Jobs zur täglichen Verschleißprognose kritischer Assets (Primärbrecher FLSmidth, Zyklonpumpen Weir Minerals).
    \item Historische Analysen zur Anpassung von Wartungsstrategien in SAP, mit Fokus auf condition-based statt statischer Intervalle.
  \item Optimierung von SQL Stored Procedures zur Beschleunigung der Datenaufbereitung für ML-Training und statistische Modellierung.\\[-1em]
  \end{itemize}

  \textbf{Wartungsverlässlichkeit – Junior Data Analyst 1}\\[-1em]
  \begin{itemize}[nosep,leftmargin=1em]
    \item Transformation von Wartungsintervallen durch fortgeschrittene Zeitreihenanalysen (AR-, MA- und ARIMA-Modelle) zur Vorhersage von Verschleiß- und Ausfallmustern.
    \item Automatisierung von KPI-Reportings mit Dagster-Jobs und Entwicklung von Power-BI-Dashboards zur schnellen Bereitstellung und Überwachung von Maschinenzuständen.
    \item Optimierung von SQL-Abfragen für schnellere Datenverarbeitung.\\[-1em]
  \end{itemize}

  \textbf{Wartungsverlässlichkeit – Trainee Data Analyst}\\[-1em]
  \begin{itemize}[nosep,leftmargin=1em]
      \item Langzeit-Analysen der Anlagenperformance (Primärbrecher, Förderbänder, Zyklonpumpen, Kugelmühlen, Sekundärbrecher, HPGR) durch Verknüpfung von Musteranomalien, RCA-Ergebnissen und Betriebsvariablen zur zustandsbasierten Klassifikation von Ausfällen.
  \item Erstellung von Berichten und Dashboards in Power BI zur Optimierung von Wartungsstrategien und Steigerung der Anlagenverfügbarkeit.
  \end{itemize}
\end{expblock}
\end{minipage}
\\[-0.2em]

% ===== IMCO =====
\datecell[-1.6ex]{\small 08.2020 – 03.2021} &
\begin{minipage}[t]{\linewidth}\vspace{0pt}
\begin{expblock}
  \footnotesize
  \setlength{\parindent}{0pt}\setlength{\parskip}{0pt}
  \textbf{\href{https://www.linkedin.com/company/imco-servicios-s-a-c/posts/?feedView=all}{IMCO Servicios S.A.C, Arequipa, Perú}}\\[-0.1em]
  \textbf{Technik und Entwicklung – Junior Engineer}\\[-1em]
  \begin{itemize}[nosep,leftmargin=1em]
    \item Durchführung von statischen \& dynamischen Strukturanalysen mit Topologieoptimierung im Rahmen einer BIM-orientierten Methodik, unter Einsatz von Tekla für Architekturmodellierung sowie FEM-basiertem Strukturdesign (SAP2000, IdeaStatica, Inventor, Ansys Structural) und abschließender Planprüfung in AutoCAD 3D.
  \end{itemize}
\end{expblock}
\end{minipage}

\\[-1.5em]

% ===== Freeport Praktikum =====
\datecell[-1.6ex]{\small 01.2020 – 06.2020} &
\begin{minipage}[t]{\linewidth}\vspace{0pt}
\begin{expblock}
  \footnotesize
  \setlength{\parindent}{0pt}\setlength{\parskip}{0pt}
    \textbf{\href{https://www.fcx.com}{Freeport-McMoRan Cerro Verde, Arequipa, Perú}}\\[-0.1em]
  \textbf{Wartungszuverlässigkeit – Praktikant}\\[-1em]
  \begin{itemize}[nosep,leftmargin=1em]
    \item Einsatz von Mehrphysik-CFD-Simulation (Ansys Fluent, Autodesk CFD) zur Vorhersage von Verschleißmechanismen und zur Bewertung von Schutzschichten in kritischen Komponenten.  
    \item Beitrag zur Optimierung von Wartungsstrategien durch simulationsgestützte Entscheidungsfindung und Lebensdauerprognose.  
  \end{itemize}
\end{expblock}
\end{minipage}

\end{tabular}