% Definir longitudes para una sola columna
\newlength{\AwardDate}
\newlength{\AwardDesc}
\newlength{\AwardCert}

\setlength{\AwardDate}{1.4cm}   % columna fecha
\setlength{\AwardCert}{0.4cm}   % columna certificado [Z]
\setlength{\AwardDesc}{\dimexpr(\textwidth - \AwardDate - \AwardCert - 1.5em)\relax}

\noindent
\begin{tabular}{@{}%
  >{\raggedleft\arraybackslash}m{\AwardDate}%
  @{\hspace{0.4em}}%
  >{\RaggedRight\arraybackslash}m{\AwardDesc}%
  @{\hspace{0.4em}}%
  >{\centering\arraybackslash}m{\AwardCert}@{}}


% Premio 1 — Top Student
\footnotesize 2019 & 
\footnotesize \textbf{Bester Absolvent (Top 1 \%)} – Maschinenbau, Universidad Nacional de San Agustín, Peru. &
\cert{https://link-carta-2019} \\[0.2em]

% Premio 2 — Outstanding Thesis
\footnotesize 2021 & 
\footnotesize \textbf{Auszeichnung für herausragende Bachelorarbeit} – Ingenieurwissenschaften. &
\cert{https://link-carta-2021} \\[0.2em]

% Premio 3 — Triple Award Freeport
\footnotesize 2022 & 
\footnotesize \textbf{Innova 2022 – Dreifach Preisträger} (1. Platz Digital Transformation, Showroom \& President’s Award) – Freeport-McMoRan.  
\emph{Machine \& Deep Learning Life Data Analysis}. &
\cert{https://link-carta-2022} 
\\[-0.5em]

\multicolumn{3}{@{}r@{}}{\tiny \textcolor{linksoft}{\textit{Hinweis:} \tiny[Z] Zertifikat / Brief}} \\

\end{tabular}
