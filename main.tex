\documentclass[a4paper,11pt]{article}
\usepackage[top=1.5cm, bottom=0cm, left=2cm, right=2cm]{geometry}

\usepackage{graphicx}
% \usepackage{url}
% \usepackage{palatino}
\usepackage{fontspec}
\setmainfont{TeX Gyre Pagella}

\usepackage{array}
\usepackage{ragged2e}        % necesario para \RaggedRight en \newcolumntype
\usepackage{tabularx}
\usepackage{multirow}
\usepackage{enumitem}

% \usepackage{color}
\usepackage{xcolor}
\usepackage{hyperref}
\usepackage{yaac-icons}
\usepackage[table]{xcolor}
\usepackage{amssymb}
\usepackage{booktabs}


\usepackage{multirow}

% \newcolumntype{L}[1]{>{\RaggedRight\arraybackslash}p{#1}}

\newcommand{\topbottomstrut}{\rule{0pt}{1.1em}} % “strut” para altura interna
\setlength{\extrarowheight}{0pt}
\renewcommand\arraystretch{1}

\usepackage[framemethod=tikz]{mdframed} % dibuja marcos finos y limpios

% \textheight=9.75in
\raggedbottom
\setlength{\tabcolsep}{0in}
\renewcommand{\labelitemii}{$\circ$}
\setlength{\parindent}{0pt}
\setlength{\tabcolsep}{0pt}
\renewcommand\arraystretch{1}

\usepackage{array,graphicx,multirow,ragged2e}
\usepackage[table]{xcolor} % lädt colortbl mit

\definecolor{background}{RGB}{216,228,236}
\definecolor{mygrey}{gray}{0.75}
\definecolor{myblue}{RGB}{0,40,70}
\definecolor{lightgrey}{RGB}{240,240,240}
\definecolor{accent}{RGB}{0,140,200} % azul acento para el degradado/bandas finas
\definecolor{qrblue}{RGB}{0,42,68} 
\definecolor{cvbackgroundgray}{RGB}{235,235,235}
\definecolor{linksoft}{RGB}{30,100,160} % azul más claro
\definecolor{linksoft2}{RGB}{100,160,220}

\colorlet{symbolcolor}{linksoft}
\colorlet{linkcolor}{linksoft}

\hypersetup{
  colorlinks = true,
  linkcolor  = linksoft,
  urlcolor   = linksoft,
  citecolor  = linksoft,
}

\usepackage{fancyhdr}
\usepackage{lastpage} % solo si quieres “Página X de Y”

\pagestyle{empty}


\newmdenv[
  linecolor=mygrey,           % o black
  linewidth=0.6pt,
  leftline=true,
  rightline=false, topline=false, bottomline=false,
  innerleftmargin=8pt,
  innerrightmargin=0pt,
  innertopmargin=0pt,         % <-- sin acolchado arriba
  innerbottommargin=0pt,
  skipabove=0pt, skipbelow=0pt
]{expblock}


% --- Anchos acoplados a \textwidth ---
\newlength{\datecol}
\setlength{\datecol}{2.4cm}   % ancho de la columna de fechas (ajústalo si quieres)

\newlength{\colgap}
\setlength{\colgap}{6mm}      % separación visual entre fecha y contenido

\newcolumntype{L}[1]{>{\RaggedRight\arraybackslash}p{#1}} % texto pegado izquierda
\newcolumntype{R}[1]{>{\RaggedLeft\arraybackslash}p{#1}} 



% fecha con ajuste vertical opcional: \datecell[<desplazamiento>]{<texto>}
\newcommand{\datecell}[2][0pt]{%
  \raisebox{#1}{%
    \parbox[t]{\datecol}{\raggedleft\textbf{#2}\strut}%
  }%
}

% En el preámbulo (para viñetas cuadradas más pequeñas)
%\renewcommand{\labelitemi}{\tiny$\blacksquare$}



\usepackage{tikz}
\usetikzlibrary{calc,shadows.blur,decorations.pathmorphing,fadings}
\usetikzlibrary{shadings}
\usepackage{eso-pic}      % para dibujar sobre la página
\usepackage{qrcode}       % para generar el código QR


\usepackage{needspace} % para reservar espacio y evitar saltos


\newcommand{\isep}{-2 pt}
\newcommand{\lsep}{-0.5cm}
\newcommand{\lsepe}{-0.4cm}
\newcommand{\psep}{-0.6cm}
\newcommand{\resitem}[1]{\item #1 \vspace{-2pt}}

\newcommand{\sq}[1]{\raisebox{0.25ex}{\rule{#1}{#1}}}

% Tamaño recomendado: 0.45ex–0.70ex (prueba y elige)
\setlist[itemize]{%
  label=\sq{0.65ex},   % tamaño de la viñeta (más pequeña que 0.75ex)
  labelsep=0.4em,     % espacio viñeta–texto (reduce si quieres aún más: 0.3em)
  leftmargin=0.5em,    % sangría del bloque (reduce “espacio antes” de la viñeta)
  labelindent=0pt,     % empuje extra de la viñeta (déjalo 0 salvo que quieras mover solo el símbolo)
  itemsep=1pt,         % espacio entre ítems
  parsep=0pt,          % sin espacio entre párrafos internos
  topsep=0.05pt,          % espacio antes y después de la lista
  partopsep=0pt        % sin extra por empezar párrafo
}



% --- en el preámbulo ---
\newlength{\framepad}
\setlength{\framepad}{8pt} % el mismo padding que quieres para ambas cajas

% helper para cajas a todo el ancho, perfectamente alineadas
\newcommand{\FullColorBox}[2]{%
  \begingroup
  \setlength{\fboxsep}{\framepad}%
  \noindent\colorbox{#1}{%
    \parbox{\dimexpr\textwidth\fboxsep\relax}{#2}%
  }%
  \endgroup
}

% --- Definición de resheading ---
\newcommand{\resheading}[1]{%
  \begingroup
  \setlength{\fboxsep}{6pt}%
  \noindent\colorbox{background}{%
    \parbox{\dimexpr\linewidth-0.01\fboxsep\relax}{%
      \footnotesize\textbf{\textcolor{myblue}{#1}}%
    }%
  }%
  \endgroup
}


\newlength{\myPad}
\setlength{\myPad}{0pt} % padding que quieras

% Caja a todo el ancho con relleno
\newcommand{\fullpadbox}[2]{% #1=color, #2=contenido
  \begingroup
  \setlength{\fboxsep}{\myPad}% controla el padding
  \noindent
  \colorbox{#1}{%
    \parbox{\dimexpr\linewidth+3\fboxsep\relax}{#2}%
  }%
  \endgroup
}


% --- Topbar limpio, angosto y con QR a juego ---
\newcommand{\FancyTopbar}{%
  \begin{tikzpicture}[remember picture,overlay]
    % Alturas: más angosto y pegado arriba
    \def\hOne{6mm}      % altura franja principal (antes 8mm)
    \def\hTwo{12mm}     % alcance piezas diagonales (antes 16mm)
    \def\overshoot{0.8pt} % evita micro-hueco en el borde superior

    % Color del QR (ajústalo si quieres más contraste)
    \colorlet{qrcolor}{myblue!15!white}

    % Franja sólida superior (con overshoot)
    \fill[myblue]
      ($(current page.north west)+(0,\overshoot)$) rectangle
      ($(current page.north east)+(0,-\hOne)$);

    % Pieza diagonal clara (efecto geométrico, sin líneas)
    \path[fill=background]
      ($(current page.north east)+(0,-\hOne)$) --
      ($(current page.north east)+(-6.2cm,-\hOne)$) --
      ($(current page.north east)+(-4.6cm,-\hTwo)$) --
      ($(current page.north east)+(0,-\hTwo)$) -- cycle;

    % Transición sutil (puedes eliminarla si quieres 100% plano)
    \path[shade, left color=accent, right color=myblue, opacity=0.85]
      ($(current page.north east)+(-6.2cm,-\hOne)$) --
      ($(current page.north east)+(-10.5cm,-\hOne)$) --
      ($(current page.north east)+(-9.0cm,-\hTwo)$) --
      ($(current page.north east)+(-4.6cm,-\hTwo)$) -- cycle;


  \end{tikzpicture}%
}

\newcommand{\FancyQRBottom}{%
  \begin{tikzpicture}[remember picture,overlay]
    % color del QR: en sintonía con tus enlaces (linksoft) o con myblue
    \colorlet{qrcolor}{linksoft} % o: \colorlet{qrcolor}{myblue!20!white}

    % Nodo en la esquina inferior derecha, con pequeño margen interior
    \node[
      anchor=south east,
      inner sep=1mm,
      fill=white,
      draw=qrcolor,
      line width=0.45pt,
      rounded corners=1mm
    ] at ($(current page.south east)+(-4mm,4mm)$) {%
      \begingroup
        \color{qrcolor}
        \qrcode[height=10mm]{https://manuelcaipocc.github.io/cv-mcaipocc/}%
      \endgroup
    };
  \end{tikzpicture}%
}

\makeatletter
\AddToHook{shipout/foreground}{%
  \begin{tikzpicture}[remember picture,overlay]
    % Ajusta el + (0,8mm) si lo quieres más alto/bajo
    \node[anchor=south, inner sep=0pt] 
      at ($(current page.south)+(0,8mm)$)
      {\footnotesize \textcolor{myblue}{\thepage}};
  \end{tikzpicture}%
}

% ===== Macro para el enlace de certificado =====
\newcommand{\cert}[1]{\raisebox{0.30ex}{\href{#1}{\tiny [Z]}}}

% ===== Macro para instituciones con enlace =====
\newcommand{\inst}[2]{\href{#1}{#2}}
% --- Aplica en todas las páginas (o usa shipout/firstpage si solo la primera) ---
\AddToHook{shipout/foreground}{\FancyTopbar}
\AddToHook{shipout/foreground}{\FancyQRBottom}

\makeatother
% ----------------------------------------------------------------

\begin{document}

\setlength{\tabcolsep}{6pt}
\begin{tabular}{L{0.7\linewidth} R{0.275\linewidth}}
  
  \vspace{-8em}
  {\Large\bfseries Manuel Caipo}

  
  \vspace{0.5em}
  \textbf{Data Scientist \,|\, Maschinenbauingenieur}

  \vspace{0.5em}
  % ==== Datos personales ====
  \footnotesize
  15.02.97, Perú \,|\, Schubartstraße 62 \,|\, 89134 Blaustein\par
  \href{mailto:manuelcaipocc@outlook.com}{manuelcaipocc@outlook.com} \,|\, \smartphone{+49 176 32618095}\par
  \github{https://github.com/manuelcaipocc}\,|\, 
  \linkedin{https://www.linkedin.com/in/manuel-caipo-89845b151}\,|\, 
  \personalwebsite{https://manuelcaipocc.github.io/cv-mcaipocc/}\,|\, 
  \workpermit{https://manuelcaipocc.github.io/cv-latex-caipo/cv_links/ID/Aufenthaltstitel_caipo.pdf}


\setlength{\tabcolsep}{3pt}
\begin{tabular}{@{}l l l@{}}
Deutsch: C1 \,[\href{https://1drv.ms/b/s!AlDxETe5fFgfo9lgpP0q0_ITTcZ6oA?e=j4ouwR}{Zertifikat}] &
Englisch: C1 &
Spanisch: Muttersprache
\end{tabular}

&
  \includegraphics[width=0.5\linewidth]{images/manuelcaipo.jpg}
\end{tabular}

\vspace{-0.8em}

% ===================== ZUSAMMENFASSUNG (ANCHO COMPLETO) =====================

\begin{tabular}{p{\linewidth}}
  \footnotesize
  \justifying
    \cellcolor{cvbackgroundgray}
     \noindent Data Scientist \& Data Engineer mit Schwerpunkt auf AI-gestützten Lösungen, 
    Predictive Maintenance und skalierbaren Datenarchitekturen. 
    Nachweisbarer Business-Impact durch messbare Verbesserungen zentraler KPIs 
    wie Anlagenverfügbarkeit, Produktionsoutput und OEE. 
    Umfassende Erfahrung im Handling von Time-Series-Daten, im Aufbau und der 
    Automatisierung von ETL-Pipelines sowie in der Entwicklung, Validierung und 
    dem Deployment von ML/DL-Modellen. 
    Starke Integration von IoT-Datenströmen zwischen Shopfloor, Cloud und 
    Analytics-Schichten (Azure ML, Snowflake, Databricks, OPC UA, MQTT). 
    Praxisorientierter Maschinenbau-Hintergrund mit Fokus auf datengetriebene 
    Optimierung industrieller Systeme; sofort einsatzbereit für 
    anspruchsvolle Vollzeitrollen.
\end{tabular}

\vspace{0.2em}

% ======= EXPERIENCIA (mismo ancho que \textwidth) =======

\resheading{BERUFSERFAHRUNG}

\vspace{-1em}
% \vspace{2em}
\noindent
% --- Anchos acoplados a \textwidth ---
\newlength{\datecol}
\setlength{\datecol}{2.7cm}   % ancho de la columna de fechas (ajústalo si quieres)

\newlength{\colgap}
\setlength{\colgap}{6mm}      % separación visual entre fecha y contenido


\begin{tabular}{@{}>{\raggedleft\arraybackslash}m{\datecol} @{\hspace{\colgap}} p{\dimexpr\textwidth-\datecol-\colgap\relax}@{}}


% ===== Bosch =====
\datecell[-1.6ex]{\small 11.2024 – Heute} &
\begin{minipage}[t]{\linewidth}\vspace{0pt}
\begin{expblock}
  \footnotesize
  \setlength{\parindent}{0pt}\setlength{\parskip}{0pt}
  \textbf{\href{https://www.boschrexroth.com/en/dc/}{Bosch Rexroth, Ulm, Deutschland}}\\[-0.1em]
  \textbf{SSD3 – Data Scientist Werkstudent}\\[-1em]
  \begin{itemize}[nosep,leftmargin=1em]
    \item Aufbau und Orchestrierung von ETL-Pipelines für hochfrequente Zeitreihendaten (SQL \& NoSQL).
    \item Integration von Message-Brokern (Solace, MQTT) zur Optimierung industrieller Datenflüsse.
    \item Entwicklung von ML-/DL-basierten Prognosemodellen zur Anomalieerkennung \& RUL-Vorhersage in Hydraulikanlagen.
    \item Implementierung erklärbarer KI-Methoden zur Steigerung von Transparenz \& Akzeptanz.
  \end{itemize}
\end{expblock}
\end{minipage}
\\[-0.2em]


% ===== Freeport =====
\datecell[-1.6ex]{\small 04.2021 – 07.2023} &
\begin{minipage}[t]{\linewidth}\vspace{0pt}
\begin{expblock}
  \footnotesize
  \setlength{\parindent}{0pt}\setlength{\parskip}{0pt}
  \textbf{\href{https://www.fcx.com}{Freeport-McMoRan Cerro Verde, Phoenix, U.S.A.}}\\[-0.1em]
  \textbf{Data Analytics – Junior Data Scientist 2}\\[-1em]
\begin{itemize}[nosep,leftmargin=1em]
  \item Entwicklung und Skalierung von ML/DL-Modellen (LSTM, CNN, XGBoost) in Azure ML für Anomalieerkennung und Verschleißprognosen; weltweiter Einsatz in mehreren Minen.
  \item Implementierung von Explainability (SHAP, PDP) und kontinuierlichen Retraining-Prozessen zur transparenten, adaptiven Modellnutzung.
  \item Einführung von RUL-Modellen zur proaktiven Wartungsplanung; +1,5 \% Anlagendisponibilität (≈10 Mio. USD/Jahr Einsparung). Ausgezeichnet mit Innova 2022, Showroom Award und President’s Award.
\end{itemize}


  \textbf{Data Analytics – Junior Data Scientist 1}\\[-1em]
  \begin{itemize}[nosep,leftmargin=1em]
    \item Implementierung von ML-Modellen in Azure ML Jobs zur täglichen Verschleißprognose kritischer Assets (Primärbrecher FLSmidth, Zyklonpumpen Weir Minerals).
    \item Historische Analysen zur Anpassung von Wartungsstrategien in SAP, mit Fokus auf condition-based statt statischer Intervalle.
  \item Optimierung von SQL Stored Procedures zur Beschleunigung der Datenaufbereitung für ML-Training und statistische Modellierung.\\[-1em]
  \end{itemize}

  \textbf{Wartungsverlässlichkeit – Junior Data Analyst 1}\\[-1em]
  \begin{itemize}[nosep,leftmargin=1em]
    \item Transformation von Wartungsintervallen durch fortgeschrittene Zeitreihenanalysen (AR-, MA- und ARIMA-Modelle) zur Vorhersage von Verschleiß- und Ausfallmustern.
    \item Automatisierung von KPI-Reportings mit Dagster-Jobs und Entwicklung von Power-BI-Dashboards zur schnellen Bereitstellung und Überwachung von Maschinenzuständen.
    \item Optimierung von SQL-Abfragen für schnellere Datenverarbeitung.\\[-1em]
  \end{itemize}

  \textbf{Wartungsverlässlichkeit – Trainee Data Analyst}\\[-1em]
  \begin{itemize}[nosep,leftmargin=1em]
      \item Langzeit-Analysen der Anlagenperformance (Primärbrecher, Förderbänder, Zyklonpumpen, Kugelmühlen, Sekundärbrecher, HPGR) durch Verknüpfung von Musteranomalien, RCA-Ergebnissen und Betriebsvariablen zur zustandsbasierten Klassifikation von Ausfällen.
  \item Erstellung von Berichten und Dashboards in Power BI zur Optimierung von Wartungsstrategien und Steigerung der Anlagenverfügbarkeit.
  \end{itemize}
\end{expblock}
\end{minipage}
\\[-0.2em]

% ===== IMCO =====
\datecell[-1.6ex]{\small 08.2020 – 03.2021} &
\begin{minipage}[t]{\linewidth}\vspace{0pt}
\begin{expblock}
  \footnotesize
  \setlength{\parindent}{0pt}\setlength{\parskip}{0pt}
  \textbf{\href{https://www.linkedin.com/company/imco-servicios-s-a-c/posts/?feedView=all}{IMCO Servicios S.A.C, Arequipa, Perú}}\\[-0.1em]
  \textbf{Technik und Entwicklung – Junior Engineer}\\[-1em]
  \begin{itemize}[nosep,leftmargin=1em]
    \item Durchführung von statischen \& dynamischen Strukturanalysen mit Topologieoptimierung im Rahmen einer BIM-orientierten Methodik, unter Einsatz von Tekla für Architekturmodellierung sowie FEM-basiertem Strukturdesign (SAP2000, IdeaStatica, Inventor, Ansys Structural) und abschließender Planprüfung in AutoCAD 3D.
  \end{itemize}
\end{expblock}
\end{minipage}

\\[-1.5em]

% ===== Freeport Praktikum =====
\datecell[-1.6ex]{\small 01.2020 – 06.2020} &
\begin{minipage}[t]{\linewidth}\vspace{0pt}
\begin{expblock}
  \footnotesize
  \setlength{\parindent}{0pt}\setlength{\parskip}{0pt}
    \textbf{\href{https://www.fcx.com}{Freeport-McMoRan Cerro Verde, Arequipa, Perú}}\\[-0.1em]
  \textbf{Wartungszuverlässigkeit – Praktikant}\\[-1em]
  \begin{itemize}[nosep,leftmargin=1em]
    \item Einsatz von Mehrphysik-CFD-Simulation (Ansys Fluent, Autodesk CFD) zur Vorhersage von Verschleißmechanismen und zur Bewertung von Schutzschichten in kritischen Komponenten.  
    \item Beitrag zur Optimierung von Wartungsstrategien durch simulationsgestützte Entscheidungsfindung und Lebensdauerprognose.  
  \end{itemize}
\end{expblock}
\end{minipage}

\end{tabular}

\newpage


\resheading{TECHNISCHE FÄHIGKEITEN}
\vspace{0.2em}

{\scriptsize
\begin{tabularx}{\textwidth}{@{}X X@{}}
% ====== COLUMNA IZQUIERDA ======
\begin{minipage}[t]{\linewidth}
\textbf{\textcolor{linksoft}{Programmiersprachen}}: 
Python (Pandas, NumPy, PyTorch, scikit-learn, SciPy, SymPy, python-control), SQL, C, MATLAB \par
\vspace{0.5em}

\textbf{\textcolor{linksoft}{Machine Learning / Data Science}}: 
LSTM mit Attention, CNN, XGBoost, Random Forest, Regressionen, Clustering, Survival Models, Explainability (SHAP, PDP) \par \vspace{0.5em}

\textbf{\textcolor{linksoft}{Datenbanken / ETL}}: 
PostgreSQL, Snowflake, InfluxDB (SQL \& NoSQL), komplexe Abfragen, Stored Procedures, Dagster, Apache Airflow (Echtzeit-ETL)
\end{minipage}
&
% ====== COLUMNA DERECHA ======
\begin{minipage}[t]{\linewidth}
\textbf{\textcolor{linksoft}{Cloud \& Big Data}}: 
Azure ML, Databricks, Snowflake, Apache Kafka, Apache Flink, Solace, MQTT \par \vspace{0.5em}

\textbf{\textcolor{linksoft}{DevOps / Tools}}: 
Git, Docker, Multi-Container-Umgebungen, CI/CD \par \vspace{0.5em}

\textbf{\textcolor{linksoft}{Automatisierung / Robotik}}: 
Siemens TIA Portal (PLC), Bosch Rexroth CtrlX Core, ROS, ROS2
 \par \vspace{0.5em}

\textbf{\textcolor{linksoft}{Simulation / Modellierung}}: 
FEM (Ansys, Nastran), CFD, python-control
\end{minipage}
\\
\end{tabularx}
}


\vspace{0.2em}

% ========= EDUCATION =========
\resheading{BILDUNGSWEG}\\[\lsepe]

\vspace{1em}



% ===== Definir longitudes =====
\newlength{\colA}
\newlength{\colB}
\newlength{\colC}
\newlength{\colD}

\setlength{\colA}{3.0cm}  % Fecha
\setlength{\colD}{0.8cm}  % Columna para [i], [ii], etc.
\setlength{\colC}{1.5cm}  % Nota final (calificación)
\setlength{\colB}{\dimexpr\textwidth - \colA - \colC - \colD - 3em\relax} % Contenido principal

% ===== TABLA =====
\noindent
\begin{tabular}{@{}%
  >{\raggedleft\arraybackslash}m{\colA}      % Fecha (vertical centrado, horizontal derecha)
  @{\hspace{1em}}%
  >{\raggedleft\arraybackslash}m{\colD}      % [i], [ii] (vertical centrado, horizontal derecha)
  @{\hspace{0.3em}}%
  >{\RaggedRight\arraybackslash}m{\colB}     % Contenido (vertical centrado, horizontal izquierda)
  @{\hspace{1em}}%
  >{\raggedleft\arraybackslash}m{\colC}      % Nota (vertical centrado, horizontal derecha)
@{}}


% ---------- MSc 1 ----------
\footnotesize 10.2025 -- 10.2027 &
\raisebox{0.28ex}{\href{https://1drv.ms/b/c/1f587cb93711f150/EXqNaRoJEcxGg1Ssoym3f5oBPomNs0OYSGMpemtomssudg?e=iQAn2j}{\tiny[i]}} &
\footnotesize \textbf{Master of Science – Computational Science and Engineering} &
\footnotesize — \\[-0.15em]
& 
&
\footnotesize \textbf{ \href{https://www.uni-ulm.de/en/study/study-at-ulm-university/study-programmes/course-information/course/computational-science-and-engineering-master/}{Universität Ulm, Ulm, Deutschland }}& \\[-0.15em]

% ---------- MSc 2 ----------
\footnotesize 09.2024 -- 03.2026 &
\raisebox{0.28ex}{\href{https://1drv.ms/b/c/1f587cb93711f150/EQC2B-jFlY9Eu1x8Zm8PLtYBnmYR_lb2w23_AJ5k5wpuOA?e=jYIp3y}{\tiny[i]}} &
\footnotesize \textbf{Master of Science – Advanced Precision Engineering} &
\footnotesize 1,8\\[-0.15em] &  &
\footnotesize \textbf{\href{https://www.hs-furtwangen.de/studium/studiengaenge/advanced-precision-engineering-master}{Hochschule Furtwangen, Villingen-Schwenningen, Deutschland}}& \\[-0.15em]&
\raisebox{1.55ex} {\href{https://1drv.ms/b/c/1f587cb93711f150/EUkky3o-A8dGrnUxgqohNXABtPG748axLzTbyJIIqouM3w?e=SdTAbK}{\tiny[ii]}} &
\footnotesize Thesis: Fluid 4.0-konformes Framework für digitale Darstellung und graphbasierte neuronale Modellierung hydraulischer Systeme & \\[-0.5em]

% ---------- Postgrad ----------
\footnotesize 06.2021 -- 02.2022 &
\raisebox{0.28ex}{\href{https://siucarne.sunedu.gob.pe/autoridades/informacion-solicitud?Guid=137d2700-4dc5-47dd-8ae4-1192ba5a66b9\&Dato=Z2d6Q0RveUdMTTdJTWtjK2piZ1pBVXUrUjJTV1VWNnVqUFpzR3FST2dlbz0\%3D}{\tiny[i]}} &
\footnotesize \textbf{Postgraduiertendiplom – Maschinelles Lernen \& Deep Learning} &
\textcolor{mygrey}{\tiny [Top 5\,\%]}   \footnotesize 19.0/20.0\\[-0.6em]& &
\footnotesize \textbf{\href{https://dc.ucsp.edu.pe/postgrado/diplomado-machine-learning/}{Universidad Católica San Pablo, Arequipa, Peru}}& \\[-0.5em]

% ---------- Bachelor ----------
\footnotesize 03.2015 -- 12.2019 &
\raisebox{0.28ex}{\href{https://siucarne.sunedu.gob.pe/autoridades/informacion-solicitud?Guid=f37661dd-9383-4532-8e95-53ce4a8a7fc9\&Dato=aUhGL09TekErTE51eUIvb2VKTERYbE9iVjloaUdFN05xMUVkaStPS3Q0WT0\%3D}{\tiny[i]}}&
\footnotesize \textbf{Bachelor – Maschinenbau} &
\textcolor{mygrey}{\tiny [Top 1\,\%]}   \footnotesize 15.2/20.0\\[-0.7em]
& &
\footnotesize \textbf{ \href{https://www.unsa.edu.pe/}{Universidad Nacional de San Agustín, Arequipa, Peru}} & \\[-0.15em]
&\raisebox{0.28ex}{\href{https://1drv.ms/b/c/1f587cb93711f150/Efq_MbYwaGxKk5v9daxg2HcBgZ6ogaNsQzBStV41bFvpRw?e=FoGxN2}{\tiny[ii]}} &
\footnotesize \textit{Anerkennung durch die Regierung von Schwaben.} & \\
& \raisebox{0.28ex}{\href{https://1drv.ms/b/c/1f587cb93711f150/EcwlwdqOAZlAsxEgyMYivNEBw_89JX5WQIZyVk3WiAfeFA?e=4JS4rq}{\tiny[iii]}} &
\footnotesize \textit{Auszeichnung für die beste Studienförderungsnote.} & \\
&\raisebox{1.55ex}{\href{https://zenodo.org/records/17233982}{\tiny[iv]}} &
\footnotesize Thesis: Design eines 350 TPH Kupferkonzentrat-Förderbandes mit automatischem hydraulischem Spannsystem und Aufnahmetrichter & \\[-0.5em]

% ---------- Sekundarstufe ----------
\footnotesize 03.2009 -- 12.2013 &
\raisebox{0.30ex}{\href{https://1drv.ms/b/c/1f587cb93711f150/ETutcZ9SaZdJkxS0L8BvJaoBjHM5gMRmivIAAn_P8WEYPw?e=nu6XNA}{\tiny[i]}} &
\footnotesize \textbf{Sekundarstufe}&
\textcolor{mygrey}{\tiny [Top 5\,\%]}    \footnotesize 16.0/20.0 \\[-0.7em]
&
&
\footnotesize \textbf{ \href{https://salesianos.pe/inspectoria/nuestras-obras/arequipa/}{Salesianerschule Don Bosco }}& \\[-0.15em]

% ---------- Grundschule ----------
\footnotesize 03.2003 -- 12.2008 & &
\footnotesize \textbf{Grundschule} &
\footnotesize 18.0/20.0  \\[-0.2em]
& &
\footnotesize Charles-Gauss-Schule & \\

\\[-2em]
% ---------- Nota al pie ----------
\multicolumn{4}{@{}r@{}}{\tiny \vspace{0.5em}
\textcolor{linksoft}{\textit{Hinweise:} \tiny[i] Diplom-/Programmstatus ansehen.}}\\[-0.8em]
\multicolumn{4}{@{}r@{}}{\tiny 
\textcolor{linksoft}{\tiny Beide Masterstudiengänge: Unterrichtssprache Deutsch.}}\\




\end{tabular}


\resheading{AUSZEICHNUNGEN \& PREISE}

% Definir longitudes para una sola columna
\newlength{\AwardDate}
\newlength{\AwardDesc}
\newlength{\AwardCert}

\setlength{\AwardDate}{1.4cm}   % columna fecha
\setlength{\AwardCert}{0.4cm}   % columna certificado [Z]
\setlength{\AwardDesc}{\dimexpr(\textwidth - \AwardDate - \AwardCert - 1.5em)\relax}

\noindent
\begin{tabular}{@{}%
  >{\raggedleft\arraybackslash}m{\AwardDate}%
  @{\hspace{0.4em}}%
  >{\RaggedRight\arraybackslash}m{\AwardDesc}%
  @{\hspace{0.4em}}%
  >{\centering\arraybackslash}m{\AwardCert}@{}}


% Premio 1 — Top Student
\footnotesize 2019 & 
\footnotesize \textbf{Bester Absolvent (Top 1 \%)} – Maschinenbau, Universidad Nacional de San Agustín, Peru. &
\cert{https://manuelcaipocc.github.io/cv-caipo-latex/cv_links/unsa_bachelor/Beste_note_student.pdf} \\[0.2em]

% Premio 2 — Outstanding Thesis
\footnotesize 2021 & 
\footnotesize \textbf{Auszeichnung für herausragende Bachelorarbeit} – Ingenieurwissenschaften. &
\cert{https://link-carta-2021} \\[0.2em]

% Premio 3 — Triple Award Freeport
\footnotesize 2022 & 
\footnotesize \textbf{Innova 2022 – Dreifach Preisträger} (1. Platz Digital Transformation, Showroom \& President’s Award) – Freeport-McMoRan.  
\emph{Machine \& Deep Learning Life Data Analysis}. &
\cert{https://link-carta-2022} 
\\[-0.5em]

\multicolumn{3}{@{}r@{}}{\tiny \textcolor{linksoft}{\textit{Hinweis:} \tiny[Z] Zertifikat / Brief}} \\

\end{tabular}



% ====== REFERENZEN (ATS-optimiert) ======
\resheading{REFERENZEN}

% Longitudes (si no están ya definidas)
\newlength{\RefDate}
\newlength{\RefDesc}
\newlength{\RefCert}

\setlength{\RefDate}{1.4cm}   % columna fecha
\setlength{\RefCert}{0.4cm}   % columna certificado [Z]
\setlength{\RefDesc}{\dimexpr(\textwidth - \RefDate - \RefCert - 1.5em)\relax}

\noindent
\begin{tabular}{@{}%
  >{\raggedleft\arraybackslash}m{\RefDate}%
  @{\hspace{0.4em}}%
  >{\RaggedRight\arraybackslash}m{\RefDesc}%
  @{\hspace{0.4em}}%
  >{\centering\arraybackslash}m{\RefCert}@{}}

% Ref 1 — PhD Recommendation
\footnotesize 2025 & 
\footnotesize \textbf{Prof. Dr.-Ing. G. Ketterer}, Hochschule Furtwangen, Studiendekan M.Sc. \emph{Advanced Precision Engineering}.  
Empfehlung für Promotion. & 
\cert{https://link-ketterer-ref} \\[0.2em]

% Ref 2 — Career/Advanced Studies
\footnotesize 2023 & 
\footnotesize \textbf{T. Gaddie}, Freeport-McMoRan ,Data Scientist III, Metals Optimization.  
Referenz für Karriere \& weiterführende Studien. & 
\cert{https://www.fcx.com/} \\[0.2em]

% Ref 3 — Master Recommendation
\footnotesize 2021 & 
\footnotesize \textbf{Dr. L. Rodríguez B.}, UNSA , Direktor Maschinenbau \& Betreuer Bachelorarbeit.  
Empfehlung für Masterstudium. & 
\cert{https://link-rodriguez-ref}
\\[-0.5em]

\multicolumn{3}{@{}r@{}}{\tiny \textcolor{linksoft}{\textit{Hinweis:} \tiny[Z] Empfehlungsschreiben / Referenzbrief}} \\
\end{tabular}

% Unterhalb von AUSZEICHNUNGEN & PREISE
\resheading{STIPENDIEN \& FÖRDERUNGEN}

% Longitudes (si no están ya definidas)
\newlength{\RefDatea}
\newlength{\RefDesca}
\newlength{\RefCerta}

\setlength{\RefDatea}{1.4cm}   % columna fecha
\setlength{\RefCerta}{0.4cm}   % columna certificado [Z]
\setlength{\RefDesca}{\dimexpr(\textwidth - \RefDate - \RefCert - 1.5em)\relax}

\noindent
\begin{tabular}{@{}%
  >{\raggedleft\arraybackslash}m{\RefDate}%
  @{\hspace{0.4em}}%
  >{\RaggedRight\arraybackslash}m{\RefDesc}%
  @{\hspace{0.4em}}%
  >{\centering\arraybackslash}m{\RefCert}@{}}

% Stipendium 1 — Bachelor
\footnotesize 2017–2019 & 
\footnotesize \textbf{Beca Presidente de la República – Perú}.  
Vollstipendium für akademische Exzellenz (Top 1\%). & \\[0.2em]

% Stipendium 2 — SERESSA
\footnotesize 2019 & 
\footnotesize \textbf{Forschungsstipendium – SERESSA 2019}, Spanien.  
15th International School on Radiation Effects. & 
\end{tabular}


\resheading{ZUSÄTZLICHE KURSE}

% ================== LONGITUDES GLOBALES ==================
% Separadores y control de espacios
\newlength{\ColSep}      % espacio entre columnas internas
\newlength{\MidSep}      % espacio entre las dos mitades (entre las 2 celdas de la tabla externa)
\newlength{\RowTight}    % compactación vertical por fila

% NUEVO: Proporciones para cada mitad
\newlength{\LeftWidth}   % ancho total de la tabla izquierda
\newlength{\RightWidth}  % ancho total de la tabla derecha

% Anchos de columnas de la tabla IZQUIERDA (fecha, curso, certificado)
\newlength{\Ai}   % fecha (izquierda)
\newlength{\Bi}   % curso  (izquierda)
\newlength{\Ci}   % cert   (izquierda)

% Anchos de columnas de la tabla DERECHA (fecha, curso, certificado)
\newlength{\Aii}  % fecha (derecha)
\newlength{\Bii}  % curso  (derecha)
\newlength{\Cii}  % cert   (derecha)

% ================== VALORES POR DEFECTO (AJUSTA A GUSTO) ==================
\setlength{\ColSep}{0.4em}       % separación interna entre columnas
\setlength{\MidSep}{0.8em}       % separación entre las dos mitades
\setlength{\RowTight}{0.1em}     % compactación vertical por fila

% AQUÍ DEFINES LAS PROPORCIONES: izquierda 40%, derecha 60%
\setlength{\LeftWidth}{0.45\textwidth}
\setlength{\RightWidth}{\dimexpr\textwidth - \LeftWidth - \MidSep\relax}

% Anchos lado izquierdo
\setlength{\Ai}{0.9cm}
\setlength{\Ci}{0.35cm}
\setlength{\Bi}{\dimexpr \LeftWidth - \Ai - \Ci - 2\ColSep \relax}

% Anchos lado derecho
\setlength{\Aii}{\Ai}
\setlength{\Cii}{\Ci}
\setlength{\Bii}{\dimexpr \RightWidth - \Aii - \Cii - 2\ColSep \relax}

% ================== TABLA EXTERNA (1 FILA x 2 COLUMNAS) ==================
\noindent
\begin{tabular}{@{}p{\LeftWidth}@{\hspace{\MidSep}}p{\RightWidth}@{}}
% ================== CELDA IZQUIERDA ==================
\begin{minipage}[t]{\linewidth}\vspace{0pt}
  \footnotesize
  \begin{tabular}{@{}%
    >{\raggedleft\arraybackslash}m{\Ai}%
    @{\hspace{\ColSep}}%
    >{\RaggedRight\arraybackslash}m{\Bi}%
    @{\hspace{\ColSep}}%
    >{\centering\arraybackslash}m{\Ci}@{}}

  % -------- Filas lado IZQUIERDO --------
  2024 & Deutschkurse B1, B2, C1 (\inst{https://www.vh-ulm.de/}{Ulm vhs} \& \inst{https://www.berlitz.de}{Berlitz}) & \cert{https://manuelcaipocc.github.io/cv-caipo-latex/cv_links/german/testdaf-digital-certificate-scale.pdf} \\[\RowTight]
  2022 & Führung und Soft Skills (\inst{https://www.ucsp.edu.pe}{UCSP}) & \cert{https://manuelcaipocc.github.io/cv-caipo-latex/cv_links/courses/Leadership_soft_skills.pdf} \\[\RowTight]
  2022 & SQL (\inst{https://www.newhorizons.com}{New Horizons}) & \cert{https://manuelcaipocc.github.io/cv-caipo-latex/cv_links/courses/SQL_new_horizons.pdf} \\[\RowTight]
  2021 & Python für Data Science (\inst{https://www.socialdataconsulting.com}{SDC}) & \cert{https://manuelcaipocc.github.io/cv-caipo-latex/cv_links/courses/Python_data_science.pdf} \\[\RowTight]
  2021 & Hydrauliksysteme (\inst{https://www.boschrexroth.com}{Bosch Rexroth}) & \cert{https://manuelcaipocc.github.io/cv-caipo-latex/cv_links/courses/Rexroth_certificates.pdf} \\[\RowTight]
  2019 & Calculation of Pipelines (\inst{https://www.eadic.com}{EADIC}) & \cert{https://manuelcaipocc.github.io/cv-caipo-latex/cv_links/courses/Design_industrial_plants_eadic.pdf} \\[\RowTight]

  \end{tabular}
\end{minipage}
&
% ================== CELDA DERECHA ==================
\begin{minipage}[t]{\linewidth}\vspace{0pt}
  \footnotesize
  \begin{tabular}{@{}%
    >{\raggedleft\arraybackslash}m{\Aii}%
    @{\hspace{\ColSep}}%
    >{\RaggedRight\arraybackslash}m{\Bii}%
    @{\hspace{\ColSep}}%
    >{\centering\arraybackslash}m{\Cii}@{}}

  % -------- Filas lado DERECHO --------
  2020 & Struktur- \& Mehrphysik-Strömungsanalyse mit FEM (\inst{https://www.esss.co}{ESSS}) & \cert{https://example.com/zertifikat} \\[\RowTight]
  2019 & Industrial Plant Systems (\inst{https://www.tecsup.edu.pe}{TECSUP}) & \cert{https://academico-cloud.tecsup.edu.pe/pcc/\#/home/certificado?c=350995&n=AQP\%2FPCC\%2F17\%2F7778\&t=E} \\[\RowTight]
  2019 & Autodesk Nastran – FEM \& CFD (\inst{https://www.autodesk.com/de}{Autodesk}) & \cert{https://example.com/zertifikat} \\[\RowTight]
  2019 & Autodesk Inventor – CAD (\inst{https://www.autodesk.com/de}{Autodesk}) & \cert{https://example.com/zertifikat} \\[\RowTight]
  2018 & Process Improvement (\inst{https://www.proavance.com}{Proavance}) & \cert{https://example.com/zertifikat} \\[\RowTight]
      &                                                           & \\[-1em]
  \multicolumn{3}{@{}r@{}}{\tiny \textcolor{linksoft}{\textit{Hinweise:} [Z] Zertifikat}}\\
  \end{tabular}
\end{minipage}
\\ % fin de la única fila de la tabla externa
\end{tabular}






% \resheading{FIELDS OF INTEREST}\\[\lsep]
% \begin{itemize}
% \item Wireless Network and Network Security, Another one, a third one
% \end{itemize}



% \resheading{MAJOR PROJECTS AND SEMINAR}\\[\lsep]
% \begin{itemize}
% \item \textbf{Media Access Control Controlling} (Research Project) \\
% \emph{Guide: Prof. Hubert F., May'13 – till date}
%   \begin{itemize}\itemsep \isep
%     \item Objective: Performance analysis of HTTP web browsing traffic.
%     \item Performance analysis will help in comparing different MAC protocols based on different network scenarios.
%     \item Studied various papers related to different MAC protocols and now working on improving simulations.
%   \end{itemize}

% \item \textbf{Second Item} (M. Tech. Seminar) \\
% \emph{Guide: Prof. John Doe, Jan'14 – Aug'14}
%   \begin{itemize}\itemsep \isep
%     \item First subitem.
%     \item Second subitem.
%   \end{itemize}
% \end{itemize}

% \resheading{STRENGTHS}\\[\lsep]
% \begin{itemize}
% \item Positive Attitude, Social Interaction, Hardworking.
% \end{itemize}

% \resheading{INTEREST AND HOBBIES}\\[\lsep]
% \begin{itemize}
% \item Solving Puzzles.
% \item Playing Chess.
% \end{itemize}

\end{document}
